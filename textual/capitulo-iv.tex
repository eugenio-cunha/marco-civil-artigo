\section{Capítulo IV: Da Atuação do Poder Público}

% Bruno Germano
% Artigo 24

\noindent\textbf{Art. 24.} Constituem diretrizes para a atuação da União, dos Estados, do Distrito Federal e dos Municípios no desenvolvimento da internet no Brasil:

\noindent\textbf{I -} estabelecimento de mecanismos de governança multiparticipativa, transparente, colaborativa e democrática, com a participação do governo, do setor empresarial, da sociedade civil e da comunidade acadêmica;

\noindent\textbf{II -} promoção da racionalização da gestão, expansão e uso da internet, com participação do Comitê Gestor da internet no Brasil;

\noindent\textbf{III -} promoção da racionalização e da interoperabilidade tecnológica dos serviços de governo eletrônico, entre os diferentes Poderes e âmbitos da Federação, para permitir o intercâmbio de informações e a celeridade de procedimentos;

\noindent\textbf{IV -} promoção da interoperabilidade entre sistemas e terminais diversos, inclusive entre os diferentes âmbitos federativos e diversos setores da sociedade;

\noindent\textbf{V -} adoção preferencial de tecnologias, padrões e formatos abertos e livres;

\noindent\textbf{VI -} publicidade e disseminação de dados e informações públicos, de forma aberta e estruturada;

\noindent\textbf{VII -} otimização da infraestrutura das redes e estímulo à implantação de centros de armazenamento, gerenciamento e disseminação de dados no País, promovendo a qualidade técnica, a inovação e a difusão das aplicações de internet, sem prejuízo à abertura, à neutralidade e à natureza participativa;

\noindent\textbf{VIII -} desenvolvimento de ações e programas de capacitação para uso da internet;

\noindent\textbf{IX -} promoção da cultura e da cidadania; e

\noindent\textbf{X -} prestação de serviços públicos de atendimento ao cidadão de forma integrada, eficiente, simplificada e por múltiplos canais de acesso, inclusive remotos.

\begin{displayquote}
Comente a lei aqui  
\end{displayquote}


% Bruno Germano
% Artigo 25

\noindent\textbf{Art. 25.} As aplicações de internet de entes do poder público devem buscar:

\noindent\textbf{I -} compatibilidade dos serviços de governo eletrônico com diversos terminais, sistemas operacionais e aplicativos para seu acesso;

\noindent\textbf{II -} acessibilidade a todos os interessados, independentemente de suas capacidades físico-motoras, perceptivas, sensoriais, intelectuais, mentais, culturais e sociais, resguardados os aspectos de sigilo e restrições administrativas e legais;

\noindent\textbf{III -} compatibilidade tanto com a leitura humana quanto com o tratamento automatizado das informações;

\noindent\textbf{IV -} facilidade de uso dos serviços de governo eletrônico; e

\noindent\textbf{V -} fortalecimento da participação social nas políticas públicas.

\begin{displayquote}
Comente a lei aqui  
\end{displayquote}

% Bruno Germano
% Artigo 26

\noindent\textbf{Art. 26.} O cumprimento do dever constitucional do Estado na prestação da educação, em todos os níveis de ensino, inclui a capacitação, integrada a outras práticas educacionais, para o uso seguro, consciente e responsável da internet como ferramenta para o exercício da cidadania, a promoção da cultura e o desenvolvimento tecnológico.

\begin{displayquote}
Comente a lei aqui  
\end{displayquote}

% Eugenio Cunha
% Artigo 27

\noindent\textbf{Art. 27.} As iniciativas públicas de fomento à cultura digital e de promoção da internet como ferramenta social devem:

\noindent\textbf{I -} promover a inclusão digital;    

\begin{displayquote}
     A inclusão digital deve ser parte do processo de ensino de forma a promover a educação continuada.
\end{displayquote}

\noindent\textbf{II -} buscar reduzir as desigualdades, sobretudo entre as diferentes regiões do País, no acesso às tecnologias da informação e comunicação e no seu uso;

\begin{displayquote}      
    % Referidos incisos I e II estão intrinsecamente relacionados – o grande mote do Marco
    % Civil da Internet, além de prever a proteção dos dados dos usuários, é também promover
    % a inclusão social de parte da população que ainda não possui contato com a rede.
    % Fazendo isso, automaticamente as desigualdades serão reduzidas, pois, permitirá que as
    % pessoas se nivelem em termos de conhecimento tecnológico e tenham o mesmo ponto
    % de partida para se enfrentar do mercado de trabalho.
    % O Estado deverá desenvolver, implantar e manter políticas públicas de longo prazo, em
    % parceria com os provedores, para a inclusão digital dos brasileiros, principalmente as
    % minorias sociais que por si só não tem condições de ter acesso ao computador e à
    % Internet, objetivando o avanço da sociedade, dando-lhes autonomia para o
    % desenvolvimento do trabalho e da livre iniciativa.
    Defende o acesso a internet como medida para diminuir a desigualdade, mas destaca que cabe ao Governo promover medidas de inclusão social.
\end{displayquote}   

\noindent\textbf{III -} fomentar a produção e circulação de conteúdo nacional;

\begin{displayquote}    
    O Estado deverá desenvolver, implantar e manter políticas públicas de longo prazo, em
    parceria com os provedores, para a inclusão digital dos brasileiros, principalmente as
    minorias sociais que por si só não tem condições de ter acesso ao computador e à
    Internet, objetivando o avanço da sociedade, dando-lhes autonomia para o
    desenvolvimento do trabalho e da livre iniciativa.  
\end{displayquote}   

% Eugenio Cunha
% Artigo 28

\noindent\textbf{Art. 28.} O Estado deve, periodicamente, formular e fomentar estudos, bem como fixar metas,
estratégias, planos e cronogramas, referentes ao uso e desenvolvimento da internet no País.

\begin{displayquote}  
    % O Estado deve sempre se manter atualizado em relação às estratégias estabelecidas para
    % atingimento de suas metas de inclusão, proteção e desenvolvimento social.
    o Estado, no mínimo, haverá de se ocupar na formulação de estudos e planos para o amplo desenvolvimento da internet.
\end{displayquote}  
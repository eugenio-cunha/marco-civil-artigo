\section{Capítulo V: Disposições Finais}

% Eugenio Cunha
% Artigo 29

\noindent\textbf{Art. 29.} O usuário terá a opção de livre escolha na utilização de programa de computador em seu terminal para exercício do controle parental de conteúdo entendido por ele como impróprio a seus filhos menores, desde que respeitados os princípios desta Lei e da Lei no 8.069, de 13 de julho de 1990 - Estatuto da Criança e do Adolescente.

\begin{displayquote}  
    Em razão da liberdade e do dinamismo existente no âmbito digital, os pais ou
    responsáveis de menores poderão utilizar de programas especiais que salvaguardam as
    crianças e adolescentes de conteúdos impróprios existentes na internet.
\end{displayquote}  

\noindent\textbf{Parágrafo único.} Cabe ao poder público, em conjunto com os provedores de conexão e de aplicações de internet e a sociedade civil, promover a educação e fornecer informações sobre o uso dos programas de computador previstos no caput, bem como para a definição de boas práticas para a inclusão digital de crianças e adolescentes.

\begin{displayquote}
    Vide comentário art. 27.
\end{displayquote}

% Eugenio Cunha
% Artigo 30

\noindent\textbf{Art. 30.} A defesa dos interesses e dos direitos estabelecidos nesta Lei poderá ser exercida em juízo, individual ou coletivamente, na forma da lei.

\begin{displayquote}  
    Qualquer pessoa que se sentir lesada em seus direitos quando a infração for decorrente
    de violação de artigo dessa lei, poderá mover ação judicial, seja individual, seja
    coletiva, respeitando o princípio da inafastabilidade de jurisdição previsto no art. 5o
    XXXV da Constituição Federal.
\end{displayquote}  

% Eugenio Cunha
% Artigo 31

\noindent\textbf{Art. 31.} Até a entrada em vigor da lei específica prevista no § 2o do art. 19, a responsabilidade do provedor de aplicações de internet por danos decorrentes de conteúdo gerado por terceiros, quando se tratar de infração a direitos de autor ou a direitos conexos, continuará a ser disciplinada pela legislação autoral vigente aplicável na data da entrada em vigor desta Lei.

\begin{displayquote}
    Isso significa que, a partir do momento em que a lei entrar em vigor, ela revogará os
    dispositivos da legislação autoral no que diz respeito ao tema aqui abordado.
    
    Importante mencionar que a responsabilidade dos prestadores ou fornecedores de
    serviços em relação aos consumidores deve respeitar os ditames do Código de Defesa
    do Consumidor ou o Código Civil, conforme o caso.
\end{displayquote}

% Eugenio Cunha
% Artigo 32
\noindent\textbf{Art. 32.} Esta Lei entra em vigor após decorridos 60 (sessenta) dias de sua publicação oficial.
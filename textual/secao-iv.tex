\section{Seção IV: Da Requisição Judicial de Registros}

% Bruno Germano
% Artigo 22

\noindent\textbf{Art. 22.} A parte interessada poderá, com o propósito de formar conjunto probatório em processo judicial cível ou penal, em caráter incidental ou autônomo, requerer ao juiz que ordene ao responsável pela guarda o fornecimento de registros de conexão ou de registros de acesso a aplicações de internet.

\noindent\textbf{Parágrafo único.}  Sem prejuízo dos demais requisitos legais, o requerimento deverá conter, sob pena de inadmissibilidade:

\noindent\textbf{I -} fundados indícios da ocorrência do ilícito;

\noindent\textbf{II -} justificativa motivada da utilidade dos registros solicitados para fins de investigação ou instrução probatória; e

\noindent\textbf{III -} período ao qual se referem os registros.

\begin{displayquote}
Comente a lei aqui  
\end{displayquote}  

% Bruno Germano
% Artigo 23

\noindent\textbf{Art. 23.} Cabe ao juiz tomar as providências necessárias à garantia do sigilo das informações recebidas e à preservação da intimidade, da vida privada, da honra e da imagem do usuário, podendo determinar segredo de justiça, inclusive quanto aos pedidos de guarda de registro.

\begin{displayquote}
Comente a lei aqui  
\end{displayquote}